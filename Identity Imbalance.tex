\documentclass{beamer}
\usepackage{ragged2e}\justifying
\setlength{\parskip}{5pt}
\usetheme{Berlin}
\usecolortheme{Dolphin}
\title[ECO247 Presentation]{ Data Science, Team Collaboration in R-Studio Environment, and  How to Teach R Yourself?}
\subtitle{Applications in Economics}
% Multiple students name
\author[Group Four:Identity Imbalance]{\tiny Monisha Das \tiny 2019231006 \and
	MD. Nayan Rana \tiny 2019231012 \and
	Tomal Abbasi \tiny 2019231013 \and
	Ananta Vikram Nag \tiny 2019231016 \and
	Rajat Mojumder \tiny 2019231068 \and
	Aishi Roy Chowdhury \tiny 2019231074  \\ [5mm]
	\includegraphics[scale=.5]{SUSTlogo}}
\institute[Economics, SUST]{\large Department of Economics, \\Shahjalal University of Science and Technology, Bangladesh}
\date[\tiny]{\ November 29, 2022}

\begin{document}
	
	% Title page (1 slide)	
	\begin{frame}
		\maketitle
	\end{frame}
	
	% Slide Number 2
	
	\begin{frame}[t] {Outline of the Presentation}
		\begin{itemize}
			\item Objectives of the presentation.
			\item Data science and Application
			\item Popularity of R and Python for data science.
			\item Open source resources available for data scientists and take step forward programming.
			\item Team Collaboration in R and Python in R-Studio environment.
			\item Six simple rules for teaching yourself R.
			\item Job opportunities available for economics graduates with R and data science expertise.
			\item Data set presentation 
		\end{itemize}
		
	\end{frame}
	
	% Slide number 3
	
	\begin{frame}[t]{Objectives of the presentation}
		
		\begin{enumerate}
			\item To explore the applications of data science in economics research.
			\item  Why do open source language R and Python are popular in data science.
			\item  To identify the valuable resources for economics students to learn data science tools. 
			\item Importance of team collaboration in R and Python, and popularity of R in economics.
			\item To share the rules that proved helpful for R self-teaching.
			\item To explore economist's opportunities after mastering R, Python and Data Science.
		\end{enumerate}
		
		
	\end{frame}
	% Slide number 4
	\begin{frame}[t]{Data science and Application}
		
		
		\begin{enumerate}
			\item A data is a classification that specifies which type of value a variable has and what type of mathematical, relational, or logical operations can be applied to it without causing an error.
			\item  Data science combines math and statistics, specialized programming, advanced analytics, artificial intelligence (AI), and machine learning .
			\item  Data science , statistics and economics are works with data,this is the close relation.
			\item Data science is also  revolutionising  some of the sectors in which economists work: e.g. banking, finance, public policy and consulting.
			
		\end{enumerate}
		
	\end{frame}
	
	% Slide number 5
	\begin{frame}{Figures: Applications of data science in economics}
		\includegraphics[width=0.6\textwidth]{features}
	\small	\emph{Sourse: Quora}
	\end{frame}
	
	% Slide number 6
	\begin{frame}[t]{Popularity of R and Python for Data Science}
		
		
		\begin{enumerate}
			\item R and Python are the most reliable open source statistical software for data science
			\item  Easy to Learn
			\item  A primary reason that data scientists use open-source tools is the ability to modify the programs and code that the software relies on
			\item Using open-source software is especially useful for data scientists that want to work with libraries and other data science tools that are regularly updated. 
			\item Part of the open-source movement, which created a community of programmers who are committed to editing and updating programs and code. 
		\end{enumerate}
	\end{frame}
	
	% Slide number 7
	\begin{frame}[t]{Resources available to learn data science and take step for programming}
		
		
		\begin{enumerate}
			\item \textbf{Data analytics blogs and websites:} Towards Data 
			Science,KDnuggets,Simply Statistics,Data Science Central,Rbloggers,Github,Kaggle,Smart Data collective etc.
			\item \textbf {Data analytics you tube
			channels:} Statquest,Simplilearn,freecodecamp.org,Edureka,sentdex,Great 
			Learning,Ken Jee etc.
			\item \textbf{Choosing a Language:}
				 Despite being extremely powerful, \emph{Python/ R} is an easy language for a beginner to pick up, so give it a try! 
			
			\item Learning the language from available resources and try to understand the code
			\item Teaching yourself 
		\end{enumerate}
	\end{frame}
	
	% Slide number 8
	\begin{frame}[t]{Team Collaboration in R and Python}
		
		
		\begin{enumerate}
			\item Collaboration, transparency, and reproducibility across R and Python workflows are essential for bilingual teams that wish to undertake significant data research while enabling experts to work in their preferred language (s).
			\item \textbf {Adopt a Bilingual IDE}
			The RStudio IDE is perfect for this. It can Combine R and Python in a single script
			\item \textbf{Interoperable Enterprise Ecosystem}	R and Python code collaboration is made possible by RStudio Workbench, a safe, scalable, and centralized environment
			\item The uniform structure and opinionated syntax of a well-defined grammar enable standardization across code bases. Users of R and Python can utilize the tools dplyr, siuba, and ggplot2 to adopt a uniform syntax for data processing and visualization, respectively
		\end{enumerate}
	\end{frame}
	
	% Slide number 9
	\begin{frame}[t]{Rules that proved helpful for teaching yourself R}
		\scriptsize
		
		\begin{columns}[t] 
			\column{0.5\linewidth} {\textbf{Rule 1: Uses of Free Resources}}
			\begin{itemize}
				\item High-quality free resources are available online.
				\item   	Free courses like Coursera, edX, and freeCodeCamp. Free videos on YouTube and Twitch.
				\item   	Some regional and global R groups such as R-Ladies and ROpenSci offer blog posts, workshops, and resources in multiple languages.
				\item  	Blogs such as inSlieco, an r-directory website also uses as an R platform.
				\item  Many institutions offer free tutorials, workshops, and classes online. For example, Quebec Centre for Biodiversity Science (QCBS) R Workshop Series, Coding Club, EcoDataScience.
			\end{itemize}
			
			\column{0.5\linewidth} {\textbf{Rule 2:Use CRAN'S Task View}}
			\begin{itemize}
				\item CRAN is network of ftp and web servers around the world that store identical, up-to-date, versions of code and documentation for R.
				\item   Currently there are 18787 available packages in CRAN 
				\item  . CRAN Task View helps users to browse published packages by topic, including multivariate statistics, spatiotemporal analyses, meta-analyses and more.
				\item The Task View browser lists relevant packages for each topic and provide links to extended documentation of each featured package
				\item It is very useful for R beginers.
				
			\end{itemize}	
			
		\end{columns}		
		
	\end{frame}
	
	% Slide number 10
   \begin{frame}[t]{Rules that proved helpful for teaching yourself R}
		\scriptsize
		
		\begin{columns}[t] 
			\column{0.5\linewidth} {\textbf {Rule 3 : Adopt Good Practices and Be Consistent}}
				\begin{itemize}
				\item Starting from pseudocode:
				If you are new in any kind of programming  language , you may benefit from starting your project with psudocode.
				
				\item   Try direct coding : If you think you are comfortable by doing , you should write code directly 
				\item   Coding style : A coding style is a  bunch  of guidelines  which we use to write code perfectly.It is a collection of specific functions, packages and syntax strategies.
				\item  Set your working directory: Project directory in R is the folder where   we are working . Hence it’s the place (Environment) where we have to store files of our project.
				
			\end{itemize}
			
			\column{0.5\linewidth} {\textbf{Rule 4: Join the R community}}
			\begin{itemize}
				\item 	Offline Community 
				\item  	Online Community 
				\item   	R Learning Community:	The R for Data Science (R4DS) community  
				
				\item  	Social Networks  
				
				\item  	Twitter: @rstudio, @Rbloggers, @icymir, @RLadiesGlobal, @R4DS, and @rOpenSci\\
					Facebook: Ecology in R
					Online Events: the data science show “SLICED” on Twitch\\
					Blogs: Blogs of Jacqueline Nolis, Miles McBain etc.
				
			\end{itemize}	
			
		\end{columns}	
		
  \end{frame}
	
	% Slide number 11
 \begin{frame}[t]{Rules that proved helpful for teaching yourself R}
		\scriptsize
		
		\begin{columns}[t] 
			\column{0.5\linewidth} {\textbf{Rule 5: Read others’ code, and share yours}}
			\begin{itemize}
				\item Read other’s code and share own code in different platform like GitHub, GitLab, and Open Science Framework (OSF).
				\item   By reading and running other’s code from open source platform will be valuable in discovering new functions, optimizing processing speeds, and learning from experts
				\item   There are a large R community in the world which is very interactive to solve the problem, either to understand other’s code or to debug own code. 
				
			\end{itemize}
			
			\column{0.5\linewidth} {\textbf{Rule 6: Don’t box yourself in}}
			\begin{itemize}
				
				\item   There doesn't have to be the ending line of the programming 
				journey.
				\item R is a wonderful tool which is used for statistics,data manipulation,data 
				visualization
				\item   It can be the first step for learning other programming languages or fields 
				in future such as HTML,Julia,Python.
				\item  The knowledge and skills that is obtained 
				while learning R can be very beneficial for other coding ventures.
			
			\end{itemize}	
			
		\end{columns}
	
   \end{frame}
	
	
	% Slide number 12
	\begin{frame}[t]{Data Science Job Opportunities for Economist}
		
		
		\begin{enumerate}
			\item According to the \textbf{U.S. Bureau of Labor Statistics (BLS)}, data science jobs are among the fastest-growing occupations which will grow \textbf {33 percent} through 2030
			\item  Roughly \textbf{ 13 percent} of current data scientists have an Economics degree.
			\item  Jobs like
			Data Scientsts,
			Management,
			 Business analysts,
			 Financial analysts,
			 Marketing analysts,
			 Operational analysts,
			 Healthcare analysts, Data analytics, consultants 
			\item In the US, for Economist Data Scientist jobs pay \textbf {110,167 dollar/year} on average
			\item There are some popular websites which offer data scientist internships/jobs such 
			as LinkedIn,Indeed,Glassdoor,Intrerships.com,Fasthire etc from well known company like Amazon, Google
			
		\end{enumerate}
	\end{frame}
%Slide number 13
\begin{frame}[t]{Data Analysis}
The data in FERTIL2 were collected on \emph{women living} in the Republic of Botswana in 1988. 
There are \textbf{ 4360} observations of \textbf{ 27 }variables.\\
Here we choose some variables. These are
\begin{itemize}
	\item Years of education (educ)
	\item Having electricity (electric)
	\item Live in urban area (urban)
	\item Number of living children (children)
\end{itemize}
The variables \textbf{ electric} and \textbf{ urban} are binary indicators equal to one if 
the woman’s home has electricity and live in urban, and zero if not.
\end{frame}


\begin{frame}[t]{Quantitative Analysis}
	
	\begin{enumerate}
		\item The smallest and largest values of children in the sample are \textbf{ 0} and \textbf{ 13}. \textbf{Average children:2}
		
		\item Approximately\textbf{ 14 percent} of women have electricity in the home
		
		\item Average children with electricity 2 and without electricity 3.\textbf{ Clearly the electricity households have fewer children}
		\item There is no direct relationship between having electricity and woman have fewer child but may be influence of electricity devices like TV and radio are more likely to promote fewer children and more liberal thoughts.
		
	\end{enumerate}
	
	
\end{frame}

\begin{frame}[t]{Hypothesis}
	\textbf{H0:} There is no difference in average years of education due to electricity.\\
	\textbf	{H0:} There is no difference in average year of education due to urban and non-urban.\\
	\textbf{H0:} The effect of one independent variable on average education per year does not depend on the effect of the other independent variable.\\
	
	\textbf{	Ha:} There is a difference in average years of education due to electricity.\\
	\textbf{	Ha:} There is a difference in average year of education due to urban and non-urban.\\
	\textbf{	Ha:}There is an interaction effect between electricity and live in urban on average year of education.
\end{frame}
\begin{frame}[t]{Two way Anova}

	\begin{figure}
    \centering
	\includegraphics[width=1.1\textwidth]{screen}\\
  
	\end{figure}




\end{frame}
\begin{frame}[t]{Summary}
	\begin{enumerate}
		\item The p-value of electric is <2e-16 (significant), which indicates that the levels of electric are associated with significant different in education.
		\item The p-value of urban is <2e-16 (significant), which indicates that the levels of urban are associated with significant different in education.
		\item The p-value for the interaction between electric*urban is 7.06e-10(significant), which indicates that the relationships between urban and education depends on the electricity.
	\end{enumerate}
\end{frame}
	
	% Slide number 13
	\begingroup
	\setbeamertemplate{footline}{}
	\begin{frame} [c] { }
		
		\centering
		\large  Thank You! \\
		\centering
		\bigskip
		Questions and Answers?	
	\end{frame}
	\endgroup
	
	
\end{document}